%-----------------------------------------------
% Template para criação de resumos de projectos/dissertação
% jlopes AT fe.up.pt,   Fri Jul  3 11:08:59 2009
%-----------------------------------------------

\documentclass[9pt,a4paper]{extarticle}

%% English version: comment first, uncomment second
\usepackage[english]{babel}     % English
\usepackage{graphicx}           % images .png or .pdf w/ pdflatex OR .eps w/ latex
\usepackage{times}              % use Times type-1 fonts
\usepackage[utf8]{inputenc}     % 8 bits using UTF-8
\usepackage{url}                % URLs
\usepackage{multicol}           % twocolumn, etc
\usepackage{float}              % improve figures & tables floating
\usepackage[tableposition=top]{caption} % captions
%\usepackage{parskip}

%% page layout
\usepackage[a4paper,margin=30mm,noheadfoot]{geometry}

%% space between columns
\columnsep 12mm

%% headers & footers
\pagestyle{empty}

%% figure & table caption
\captionsetup{figurename=Fig.,tablename=Tab.,labelsep=endash,font=bf,skip=.5\baselineskip}

%% heading
\makeatletter
\renewcommand*{\@seccntformat}[1]{%
  \csname the#1\endcsname.\quad
}
\makeatother

%% avoid widows and orphans
\clubpenalty=300
\widowpenalty=300

\begin{document}

\title{\vspace*{-8mm}\textbf{\textsc{Android-based implementation of Eulerian Video Magnification\\for vital signs monitoring}}}
\author{\emph{Pedro Boloto Chambino}\\[2mm]
\small{Dissertation supervised by \emph{Prof.\ Luís Teixeira} and \emph{Luís Rosado}}\\
\small{at \emph{Fraunhofer Portugal AICOS}}}
\date{}
\maketitle
%no page number
\thispagestyle{empty}

\vspace*{-4mm}\noindent\rule{\textwidth}{0.4pt}\vspace*{4mm}

%% macros
\newcommand{\evm}{Eulerian Video Magnification}

\begin{multicols}{2}

\section{Motivation}\label{sec:motivation}

\evm{} is a method, recently presented at
\emph{SIGGRAPH}\footnote{\url{http://www.siggraph.org/}} 2012, capable of
revealing temporal variations in videos that are impossible to see
with the naked eye. Using this method, it is possible to visualize
the flow of blood as it fills the face~\cite{Wu2012Eulerian}.
Which provides enough information to assess the heart rate in a
contact-free way using a camera~\cite{Wu2012Eulerian,
Poh2010Non, Poh2011Advancements}.

Due to being recently proposed, the \evm{} method implementation
has not been tested in smartphones yet. Thus, Fraunhofer Portugal is
interested in testing the feasibility of implementing an
\evm{}-based method on a mobile device with the Android platform.

There has been some successful effort on the assessment of vital
signs, such as, heart rate, and breathing rate, in a contact-free
way using a webcamera~\cite{Wu2012Eulerian, Poh2010Non, Poh2011Advancements},
and even a smartphone~\cite{Vitrox2013, Philips2013}.

Other similar products, which require specialist hardware and are
thus expensive, include \emph{laser Doppler}~\cite{Ulyanov1993Pulse},
\emph{microwave Doppler radar}~\cite{Greneker1997Radar}, and
\emph{thermal imaging}~\cite{Garbey2007Contact}.

Since it is a cheaper method of assessing vital signs in a
contact-free way than the above products, this research work has
potential for advancing fields, such as, \emph{telemedicine},
\emph{personal health-care}, and \emph{ambient assisting living}.

Despite the existence of very similar products by
\emph{Philips}~\cite{Philips2013} and
\emph{ViTrox Technologies}~\cite{Vitrox2013}
to the one proposed on this research work, none of these implement
the \evm{} method.

\section{Objectives}\label{sec:objectives}

The main goal is to develop a lightweight, real-time \evm{}-based
method capable of executing on a mobile device. Which will require
performance optimizations and trade-offs will have to taken into account.

In the process, the creation of an Android application which
estimates a person's heart rate in real-time using the device's camera
will be developed.

\section{Work description}\label{sec:work}

% TODO estruturação
% TODO tecnologias utilizadas

\subsection{Eulerian Video Magnification}\label{sec:work:evm}

\subsection{Heart rate estimation}\label{sec:work:heart}

\section{Conclusions}\label{sec:conclusions}

\bibliographystyle{unsrt}  % numeric, unsorted refs
\bibliography{../myrefs}

\end{multicols}

\end{document}
