%-----------------------------------------------
% Template para criação de resumos de projectos/dissertação
% jlopes AT fe.up.pt,   Fri Jul  3 11:08:59 2009
%-----------------------------------------------

\documentclass[9pt,a4paper]{extarticle}

%% English version: comment first, uncomment second
\usepackage[portuguese]{babel}  % Portuguese
\usepackage{graphicx}           % images .png or .pdf w/ pdflatex OR .eps w/ latex
\usepackage{times}              % use Times type-1 fonts
\usepackage[utf8]{inputenc}     % 8 bits using UTF-8
\usepackage{url}                % URLs
\usepackage{multicol}           % twocolumn, etc
\usepackage{float}              % improve figures & tables floating
\usepackage[tableposition=top]{caption} % captions
\usepackage{indentfirst}        % portuguese standard for paragraphs
%\usepackage{parskip}

%% page layout
\usepackage[a4paper,margin=30mm,noheadfoot]{geometry}

%% space between columns
\columnsep 12mm

%% headers & footers
\pagestyle{empty}

%% figure & table caption
\captionsetup{figurename=Fig.,tablename=Tab.,labelsep=endash,font=bf,skip=.5\baselineskip}

%% heading
\makeatletter
\renewcommand*{\@seccntformat}[1]{%
  \csname the#1\endcsname.\quad
}
\makeatother

%% avoid widows and orphans
\clubpenalty=300
\widowpenalty=300

%% macros
\newcommand{\evm}{\emph{Eulerian Video Magnification}}


\begin{document}

\title{\vspace*{-8mm}\textbf{\textsc{Android-based implementation of Eulerian Video Magnification\\for vital signs monitoring}}}
\author{\emph{Pedro Boloto Chambino}\\[2mm]
\small{Dissertação sob a orientação do \emph{Prof.\ Luís Teixeira} e \emph{Luís Rosado}}\\
\small{na \emph{Fraunhofer Portugal AICOS}}}
\date{}
\maketitle
%no page number
\thispagestyle{empty}

\vspace*{-4mm}\noindent\rule{\textwidth}{0.4pt}\vspace*{4mm}

\begin{multicols}{2}

\section{Motivação}\label{sec:motivation}

\evm{} é um método, recentemente apresentado no
\emph{SIGGRAPH}\footnote{\url{http://www.siggraph.org/}} 2012, capaz de revelar
variações temporais em vídeos que são impossíveis de ver ao olho nu. Usando
este método é possível visualizar o fluxo sanguíneo da
face~\cite{Wu2012Eulerian}. Disponibilizando informação suficiente
para obter o batimento cardíaco através de uma câmara sem contacto existir
contacto~\cite{Wu2012Eulerian, Poh2010Non, Poh2011Advancements}.

Por ter sido recentemente proposto, uma implementação do método \evm{} ainda
não foi testada em \emph{smartphones}. Por isso, a Fraunhofer Portugal
está interessada em testar a viabilidade de uma implementação baseada
no método \evm{} para dispositivos móveis \emph{Android}.

Já houve esforços com sucesso na deteção de sinais vitais, como
batimento cardíaco, e taxa respiratória, através de câmaras
\emph{web}~\cite{Wu2012Eulerian, Poh2010Non, Poh2011Advancements}, e até
em \emph{smartphones}~\cite{Vitrox2013, Philips2013}.

Como é um método barato de obter sinais vitais sem necessidade de contacto,
este trabalho tem potencial para avançar campos, como \emph{telemedicine},
\emph{personal health-care}, e \emph{ambient assisting living}.

Apesar da existência de produtos semelhantes pela
\emph{Philips}~\cite{Philips2013} e \emph{ViTrox Technologies}~\cite{Vitrox2013}
ao proposto neste trabalho, nenhum destes implementa o método \evm{}.

\section{Objetivos}\label{sec:objectives}

O objetivo principal é desenvolver um método baseado no \evm{} que seja
leve e execute em tempo-real no dispositivo móvel. O que vai requerer
otimizações de performance e \emph{trade-offs} terão de ser considerados.

No processo, uma aplicação \emph{Android} capaz de estimar o batimento
cardíaco de uma pessoa em tempo-real usando a câmara do dispositivo
será criada.

\section{Descrição do Trabalho}\label{sec:work}

Para aumentar a velocidade de implementação e facilitar \emph{testing},
uma aplicação para \emph{desktop} foi desenvolvida primeiro, que foi
depois integrada numa aplicação \emph{Android}.

A aplicação foi escrita em \emph{C/C++} com a ajuda da biblioteca \emph{OpenCV},
uma biblioteca \emph{open-source} para processamento de imagem.
Consequentemente, a integração com a aplicação \emph{Android} foi realizada
através do \emph{Android Native Development Kit} e da \emph{framework}
\emph{Java Native Interface}.

O \emph{workflow} da aplicação começa por obter uma imagem da câmara do
dispositivo. A face de uma pessoa é detetada usando o módulo de deteção
de objetos do \emph{OpenCV} que foi previamente treinada para detetar faces
humanas. A região de interesse (ROI) da face da pessoa é alimentada ao método
\evm{} implementado para amplificar as variações de cor. A média do canal
verde do ROI é computado, para aumentar o rácio do sinal-para-baralho,
e persistido. Ao longo do tempo, os valores persistidos representam um sinal
\emph{photo-plethysmographic}~\cite{Verkruysse2008Remote} da variação do fluxo
sanguíneo oculto. O sinal é processado usando o método
\emph{detrend}~\cite{Tarvainen2002Advanced} para remover as tendências do
sinal sem distorcer a magnitude. O sinal é depois validado
como sendo um sinal do batimento cardíaco detetando os picos do sinal
de forma a verificar a sua forma e tempo. Finalmente, a estimativa do
batimento cardíaco é computado pela identificação da frequência com maior
\emph{power spectrum}.

\subsection{\evm}\label{sec:work:evm}

As implementações iniciais deste trabalho do método \evm{} foram escritas
em \emph{Java}. No entanto, porque o suporte para \emph{Java} da biblioteca
\emph{OpenCV} é ainda recente e devido a razões de performance a versão
final foi implementada em \emph{C/C++}.

A abordagem do método \evm{} combina processamento espacial e temporal para
realçar variações temporais subtis num vídeo. Primeiro, a sequência do vídeo
é decomposta em diferentes bandas de frequência espaciais. Como estas podem
exibir diferentes rácios sinal-para-baralho, estes podem ser magnificados
de forma diferente. Em geral, a pirâmide Laplaciana
completa~\cite{Burt1983Laplacian} pode ser computada. Depois, processamento
temporal é realizado em cada banda espacial de uma forma uniforme para todos
os pixeis de cada banda. Após isso, o sinal extraído é magnificado por um
fator de $\alpha$, que pode ser especificado pelo utilizador, e pode ser
atenuado automaticamente. Finalmente, o sinal magnificado é adicionado à
imagem original e a pirâmide espacial colapsada para obter o resultado final.

Neste trabalho, a amplificação da variação da cor é mais importante do que
magnificação de movimento. Por isso, o filtro espacial usado foi a computação
de um nível da pirâmide Gaussiana, que consiste em aplicar um filtro
Gaussiano e fazer \emph{downsample} da imagem várias vezes. Contudo, como
a performance é uma prioridade e a computação de um nível da pirâmide Gaussiana
é computacionalmente cara, este passo foi alterado para apenas uma operação
de redimensionamento usando o método de interpolação \emph{AREA} da biblioteca
\emph{OpenCV}, que produz um resultado semelhante.

O filtro temporal usado para extrair os movimentos ou sinais para serem
amplificados. Assim, a escolha do filtro é dependente da aplicação. Para
magnificação de movimento, um filtro de banda larga, como o \emph{butterworth},
é preferido. Um filtro de banda estreita produz um resultado mais livre de
barulho para amplificação da cor do fluxo sanguíneo. Em~\cite{Wu2012Eulerian},
um filtro de banda ideal foi usado devido às suas frequências \emph{cutoff}
retas. Alternativamente, para uma implementação em tempo-real, filtros IIR
de baixa ordem são úteis tanto para magnificação de movimento e amplificação
de cor. Por isso, a subtração de dois filtros IIR de baixa ordem foi usada
para a implementação do \evm{} método implementado.

\subsection{Estimativa do batimento cardíaco}\label{sec:work:heart}

A estimativa do batimento cardíaco é computado usando a transformada de
\emph{Fourier}, que é uma transformação matemática capaz de converter uma
função de tempo, $f(t)$, numa nova função representando o domínio da frequência
da função original.

Para calcular o \emph{power spectrum}, o resultado da função transformada
de \emph{Fourier} é multiplicado por si próprio.

Como os valores são capturados de um vídeo, sequência de imagens, a função
de tempo é discreta, com uma taxa de frequência igual à taxa frequência do
vídeo, $FPS$.

O índice, $i$ correspondente ao máximo do \emph{power spectrum} pode ser
convertido para um valor correspondente à frequência, $F$, do sinal, usando
a equação:

\begin{equation}
  F = \frac{i * FPS}{2 N}
\end{equation}

onde $N$ é o número de valores do sinal extraído. $F$ pode ser multiplicado
por 60 para obter uma estimativa do batimento cardíaco.

\section{Conclusões}\label{sec:conclusions}

Uma aplicação \emph{Android}, com o nome \emph{Pulse}, capaz de estimar o
batimento cardíaco de uma pessoa usando o método \evm{} foi desenvolvida.

Muito esforço foi para melhorar a performance da aplicação, para que seja capaz
de executar num dispositivo \emph{Android}. Tendo um aumento na performance
de 22\% desde a implementação inicial, como sugerida por~\cite{Wu2012Eulerian},
até uma versão otimizada.

Por causa disso, a precisão das estimativas do batimento cardíaco da aplicação
foi baixa. Medições da aplicação \emph{Pulse} de 9 participantes foram
comparadas com leituras de um aparelho de tensão. A concordância entre os dois,
de acordo a análise de gráficos Bland-Altman, obteve uma média \emph{bias}
de $-12.00$ com limites de concordância de 95\% $-33.13$ e $9.13$ bpm.

\subsection{Trabalho futuro}\label{sec:future}

Tendo desenvolvido um método baseado no \evm{} que é leve e executa em
tempo-real num dispositivo \emph{Android} para amplificar variações de cor,
a performance de outras variações do método podem ser melhoradas. Levando
à utilização do método noutros tipos de dispositivos.

A aplicação implementada, \emph{Pulse}, precisa de melhorar as suas estimativas
do batimento cardíaco, e outros sinais vitais podem ser monitorizar, como
a taxa respiratória.

Para suportar algumas características da aplicação \emph{Android} implementada,
parte da biblioteca \emph{OpenCV para Android SDK}, que ainda é recente,
teve de ser modificada. Estas modificações que podem ser contribuídas de volta
para a comunidade.

\bibliographystyle{unsrt-pt}  % numeric, unsorted refs
\bibliography{../myrefs}

\end{multicols}

\end{document}
