%-----------------------------------------------
% Template para criação de resumos de projectos/dissertação
% jlopes AT fe.up.pt,   Fri Jul  3 11:08:59 2009
%-----------------------------------------------

\documentclass[9pt,a4paper]{extarticle}

%% English version: comment first, uncomment second
\usepackage[portuguese]{babel}  % Portuguese
\usepackage{graphicx}           % images .png or .pdf w/ pdflatex OR .eps w/ latex
\usepackage{times}              % use Times type-1 fonts
\usepackage[utf8]{inputenc}     % 8 bits using UTF-8
\usepackage{url}                % URLs
\usepackage{multicol}           % twocolumn, etc
\usepackage{float}              % improve figures & tables floating
\usepackage[tableposition=top]{caption} % captions
\usepackage{indentfirst}        % portuguese standard for paragraphs
%\usepackage{parskip}

%% page layout
\usepackage[a4paper,margin=30mm,noheadfoot]{geometry}

%% space between columns
\columnsep 12mm

%% headers & footers
\pagestyle{empty}

%% figure & table caption
\captionsetup{figurename=Fig.,tablename=Tab.,labelsep=endash,font=bf,skip=.5\baselineskip}

%% heading
\makeatletter
\renewcommand*{\@seccntformat}[1]{%
  \csname the#1\endcsname.\quad
}
\makeatother

%% avoid widows and orphans
\clubpenalty=300
\widowpenalty=300

\begin{document}

\title{\vspace*{-8mm}\textbf{\textsc{Android-based implementation of Eulerian Video Magnification\\for vital signs monitoring}}}
\author{\emph{Pedro Boloto Chambino}\\[2mm]
\small{Dissertação sob a orientação do \emph{Prof.\ Luís Teixeira} e \emph{Luís Rosado}}\\
\small{na \emph{Fraunhofer Portugal AICOS}}}
\date{}
\maketitle
%no page number
\thispagestyle{empty}

\vspace*{-4mm}\noindent\rule{\textwidth}{0.4pt}\vspace*{4mm}

\begin{multicols}{2}

\section{Motivação}\label{sec:motiva}

Neste documento apresentam-se alguns conselhos e instruções para a preparação dos resumos de projecto/dissertação.
Pede-se aos autores o favor de, dentro do possível, cumprirem com as instruções que são dadas, assim como com a estrutura apresentada, de forma a manter-se o mesmo aspecto em todos os resumos.
Nas sub-secções~\ref{sec:lingua} a ~\ref{sec:number} podem encontrar-se alguns detalhes sobre a formatação do documento.

Na secção ``Motivação'' deve ser apresentado o enquadramento do trabalho, dando ideia das necessidades que o mesmo cobre.

\section{Objectivos}\label{sec:goals}

A secção ``Objectivos'' deve enunciar claramente os objectivos a atingir com o trabalho de projecto/dissertação, enquadrando-os na respectiva área de actividade a que o trabalho se destina.
Por exemplo, este documento tem como objectivos:
\begin{itemize}
\item Servir de modelo/exemplo do ponto de vista da dos resumos;
\item Apresentar o aspecto gráfico que se pretende para os resumos;
\item Disponibilizar \emph{templates} a quem pretenda utilizar \LaTeX.
\end{itemize}

\section{Descrição do Trabalho}\label{sec:work}

Na secção ``Descrição do Trabalho'' (com este ou com outro nome que se julgue mais adequado) devem ser apresentadas as principais partes do trabalho, começando pela sua estruturação. Devem ser mencionadas as tecnologias utilizadas, com referências à sua interdependência e interligação dando especial ênfase aos componentes desenvolvidos pelo estudante no âmbito do trabalho em causa.

No presente documento, seguem-se 9 subsecções com instruções acerca da extensão da comunicação, margens, estilos e outras recomendações gerais acerca da elaboração da versão final dos resumos.

\section{Conclusões}\label{sec:conclui}

Espera-se que este documento possa contribuir para uma melhor qualidade dos resumos dos projectos/dissertações do MIEIC/FEUP.

Documentos que não respeitem este aspecto gráfico serão liminarmente recusados, ficando os respectivos autores em falta em relação à sua entrega.

Para as dissertações existem regras definidas noutro lado~\cite{kn:Mat93}.

\bibliographystyle{unsrt-pt}  % numeric, unsorted refs
\bibliography{../myrefs}

\end{multicols}

\end{document}
