\chapter{Approach} \label{chap:approach}

\section*{}

In this chapter, a simple description of the approach to be taken during 
this research work is presented on section~\ref{sec:approach}. Moreover, 
the solution perspective and the difficulties and problems that may arise 
are described on section~\ref{sec:solution} and~\ref{sec:difficulties},
respectively. The last section~\ref{sec:evaluation}, describes the 
evaluation process.

\section{Overview} \label{sec:approach}

% Incremental implementation of the application...

The approach or methodology taken to reach this research work objectives
is an incremental implementation of the application features or requirements.
It should be taken into consideration that this approach plan may be modified 
during the duration of the research work and is not a strict line of action.

Starting with the creation of a simple \emph{Android} application that uses
the \emph{OpenCV} library. Follows the implementation of face detection using 
the \emph{OpenCV} library, and the \evm{} method showing the resulting video 
in real-time. Then, after the implementation of a simple heart rate detection 
algorithm based on the \evm{} method, the evaluation of application will
be executed. This evaluation should then be executed every time a significant 
modification to the heart rate detection algorithm is made.

\section{Solution Perspective} \label{sec:solution}

% Explain the solution and why it was chosen...

To extract the cardiac pulse from a person's face, first, a face must be  
discovered on the input video, for that a simple \emph{OpenCV} face detection
algorithm will be used.

Then, in order to obtain the heart rate signal, 
there is the need to focus on an area to extract that signal. However,
the amplitude variation of the signal of interest is often must smaller 
than the noise inherent in the video. To enhance these subtle signals 
spatial polling can be used. Despite of that if the spatial polling 
applied is not large enough, the signal of interest will not be revealed.
Retrieved from the article~\cite{Wu2012Eulerian}, the equation~\ref{eq:noise}
gives an estimate for the size of the spatial polling need to revel the 
signal of interest at a certain noise power level.

\begin{equation} \label{eq:noise}
  S(r) = \sigma'^2 = k \frac{\sigma^2}{r^2}
\end{equation}

Where $\sigma^2$ is the noise power level, which can be estimated by using 
a technique as in~\cite{Liu2006Noise}, and $S(r)$ is the signal power of
such spatial polling filter. Thus, since the filtered noise power level,
$\sigma'^2$, is inversely proportional to $r^2$, it is possible to solve the 
equation~\ref{eq:noise} for $r$, where $k$ is a constant that depends on the shape of the spatial filter.

This area of interest can then be tracked using a simple \emph{OpenCV} 
feature tracking algorithm to deal with artifacts' motion.

In addition, the \evm{} method can be configured to amplify the color
variation as explained on section~\ref{sec:evm-color}.

The signal extracted should be recognizable as a cardiac pulse signal,
however further processing may be needed to detrend the signal. Also, 
since the pulse computation may be affected by noise, historical estimations
to reject artifacts may be implemented as in~\cite{Poh2010Non}.

To create the \emph{augmented reality effect} of the blood flow, 
the raw \evm{} method result is added to the input video.

An important part of the development is focused on the complete 
implementation of the \evm{} since the available framework is not 
destined to be used in real-time nor on smartphones.

\section{Difficulties} \label{sec:difficulties}

% Difficulties and how to overcome those...

During the course of this research work a number of possible problems may
occur which will hinder the development of the project and research. Some
of these difficulties have been predicted and a few possible solutions to 
those will be described below.

\begin{itemize}

\item
\textbf{Problem}
Noise created by smartphone and person motion, and by lighting changes.

Every small movement may heavily influence the method result since it
also amplifies motion. Thus, too much noise may obfuscate the cardiac 
pulse intended to be detected.

\textbf{Solution Perspective}
Noise reduction may be accomplished by using face detection (and feature 
tracking) to focus on a small area on the person's face that is independent
enough to not suffer large intensity variations due to the identified 
artifacts. In addition, historical estimations to reject artifacts may be
implemented and specific configuration to the \evm{} method to 
emphasize the color change as much as possible~\cite{Wu2012Eulerian} as
explained on section~\ref{sec:evm-color}.

\pagebreak

\item
\textbf{Problem}
Smartphone computing power may not be enough for real-time processing.

The \evm{} method is able to run in real-time~\cite{Wu2012Eulerian}. However, 
the lower computing power of smartphones and extra processing, such as, 
face detection and tracking, noise reduction techniques, and signal 
normalization algorithms, may cause the method to not be able to execute 
in real-time.

\textbf{Solution Perspective}
If this problem arises, a possible solution is to reduce the computing power
required, by switching to different approaches or even the removal of 
some features, such as, feature tracking.

\end{itemize}

\section{Evaluation} \label{sec:evaluation}

% The evaluation should compare results and benchmark the algorithms...

The evaluation main goal is the validation of the cardiac pulse assessed.
This may be accomplished, by comparing the signal and average of the heart 
rate detected by the application with the one captured by an heart rate monitor
or a pulse oximeter. These measurements should be done simultaneously so they 
can be correctly compared.

In addition, these measurements will be obtained in different settings, 
such as, indoor, outdoor, with face motion, with face tilting, while the 
person is speaking, during lighting variations, to verify the application 
robustness.

At the end, the application will also be compared against the one developed
by \emph{Philips} -- Vital Signs Camera. The comparison will focus on both:
average heart rate detected, and application performance.
