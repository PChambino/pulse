\chapter{Introduction} \label{chap:intro}

\section*{}

% O primeiro capítulo da dissertação deve servir para apresentar o
% enquadramento e a motivação do trabalho e para identificar e
% definir os problemas que a dissertação aborda.
% Deve resumir as metodologias utilizadas no trabalho e termina
% apresentando um breve resumo de cada um dos capítulos
% posteriores.

\evm{} is a method, recently presented at 
\emph{SIGGRAPH}\footnote{\url{http://www.siggraph.org/}} 2012, capable of 
revealing temporal variations in videos that are impossible to see 
with the naked eye. Using this method, it is possible to visualize 
the flow of blood as it fills the face~\cite{Wu2012Eulerian}. 
Which provides enough information to assess the heart rate in a 
contact-free way using a camera~\cite{Wu2012Eulerian, 
Poh2010Non, Poh2011Advancements, Philips2013}.

\section{Context} \label{sec:context}

The main field of this research work is \emph{image processing 
and computer vision}, whose main purpose is to translate dimensional 
data from the real world in the form of images into numerical 
or symbolical information.

Other fields include \emph{medical applications}, \emph{software 
development for mobile devices}, \emph{digital signal processing}.

This research work is an internal project of \emph{Fraunhofer 
Portugal}\footnote{\url{http://www.fraunhofer.pt/}} proposed by 
Luís Rosado. Fraunhofer Portugal a is non-profit private association 
founded by Fraunhofer-Gesellschaft\footnote{\url{http://www.fraunhofer.de/en/about-fraunhofer/}}~\cite{Fraunhofer2013} and

\begin{quote}
  ``aims on the creation of scientific knowledge capable of 
  generating added value to its clients and partners, exploring 
  technology innovations oriented towards economic growth, the 
  social well-being and the improvement of the quality of life of 
  its end-users.''~\cite{Fraunhofer2013}
\end{quote}

\section{Motivation} \label{sec:motivation}

Due to being recently proposed, the \evm{} method implementation 
has not been tested in smartphones yet.

There has been some successful effort on the assessment of vital 
signs, such as, heart rate, and breathing rate, in a contact-free 
way using a webcamera~\cite{Wu2012Eulerian, Poh2010Non, Poh2011Advancements}, 
and even a smartphone~\cite{Philips2013}.

Other similar products, which require specialist hardware and are 
thus expensive, include \emph{laser Doppler}~\cite{Ulyanov1993Pulse}, 
\emph{microwave Doppler radar}~\cite{Greneker1997Radar}, and 
\emph{thermal imaging}~\cite{Garbey2007Contact}.

Since it is a cheaper method of assessing vital signs in a 
contact-free way than the above products, this research work has 
potential for advancing fields, such as, \emph{telemedicine}, 
\emph{personal health-care}, and \emph{ambient assisting living}.

Despite the existence of a very similar product by 
\emph{Philips}~\cite{Philips2013} -- \emph{Vital Signs Camera} -- 
to the one proposed on this research work, the product to be developed 
during this research work will have additional features, such as, 
feature tracking while assessing the heart rate, and augmented reality 
to visualize the blood flow.

\section{Objectives} \label{sec:objectives}

% Enumera os objetivos do trabalho terminando
% com um resumo das metodologias para a prossecução dos objetivos.

This research work goal is to test the feasibility of the 
implementation of the \evm{} method on smartphones by developing
an \emph{Android} application for monitoring vital signs based on 
the \evm{} method.

The proposed application should include the following features:

\begin{itemize}
  \item heart rate detection and assessment based on the \evm{} 
        method;
  \item display real-time changes, such as, the magnified blood 
        flow, obtained from the \evm{} method;
  \item deal with artifacts' motion, due to, person and/or 
        smartphone movement.
\end{itemize}

The application performance will then be evaluated through tests
with several individuals and the assessed heart rate compared to
the one detected by the \emph{Philips} application~\cite{Philips2013}, 
and to the measurement of an heart rate monitor or a pulse 
oximeter.

In order to achieve the objectives proposed and solve any possible 
problem, a review of the state of the art in the domain of vital 
signs monitoring through image processing and analysis that can be 
applicable to smartphones will also be held during this research work.

\pagebreak

\section{Outline} \label{sec:outline}

The rest of the document is organized as follows:

\begin{description}
  \item[Chapter~\ref{chap:sota}] introduces the concepts necessary to 
        understand the presented problem. In addition, it presents
        the existing related work, and a description of the technologies 
        to be used.
  \item[Chapter~\ref{chap:approach}] presents the approach taken to 
        solve the problem. Moreover, it introduces the testing and 
        evaluation methodologies.
  \item[Chapter~\ref{chap:workplan}] presents the work to be completed 
        during the master thesis, split in main tasks, including an 
        estimated schedule.        
\end{description}
